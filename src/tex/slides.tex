%%%%%%%%%%%%%%%%%%%%%%%%%%%%%%%%%%%%%%%%%%%%%%%%%%%%%%
%% Slides : Introduction to probabilistic reasoning %%
%%%%%%%%%%%%%%%%%%%%%%%%%%%%%%%%%%%%%%%%%%%%%%%%%%%%%%

%% PREAMBLE
%% Define document class and basic options
\documentclass{beamer}
%\setlength{\parindent}{0pt}

%% Load packages
\usepackage{palatino}
\usepackage{amsfonts}
\usepackage{amsmath}
%\usepackage{url}
\usepackage{hyperref}
%\usepackage{listings}
\usepackage{verbatim}
\usepackage[utf8]{inputenc} %% For french
\usepackage{tikz} %% For drawing (eg, Venn diagrams)
\usefonttheme{serif}

\hypersetup{
	colorlinks=true,
	linkcolor=blue,
	citecolor=red,
	filecolor=blue,
	urlcolor=blue
}

\usetheme{Madrid}

%% Basic info
\title{Introduction à la pensée et aux méthodes en probabilité}
%\subtitle{}
\author{Roy Nitulescu\inst{1}}

\institute
{
    \inst{1}%
    CITADEL\\
    CR-CHUM
}

\date[UdeM, Sept. XX, 2023]{Université de Montréal, Sept. XX, 2023}

\AtBeginSection[]
{
    \begin{frame}
        \frametitle{Table des matières}
        \tableofcontents[currentsection]
    \end{frame}
}

\AtBeginSubsection[]
{
    \begin{frame}
        \frametitle{Table des matières}
        \tableofcontents[currentsubsection]
    \end{frame}
}


%% BEGIN DOCUMENT
\begin{document}

%%%%
%% Slides
%%%%

\frame{\titlepage}

\begin{frame}
    \frametitle{Épigraphe}
    ``La probabilité est le concept le plus important de la science moderne,
    d'autant plus que personne n'a la moindre idée de ce qu'elle signifie.'' -- Bertrand Russell, 1929
\end{frame}


\begin{frame}
    \frametitle{Préambule}
    
    Le code source pour cette présentation ce trouve ici:

    \vfill

    \url{https://github.com/rnitulescu/prob2023}
\end{frame}


\begin{frame}
    \frametitle{Table des matières}
    \tableofcontents
\end{frame}


%%%%
%% Le raisonnement
%%%%

\section{Le raisonnement}

\begin{frame}
    \frametitle{Le raisonnement}
    
\end{frame}


\begin{frame}
    \frametitle{Le raisonnement par déduction}
    
\end{frame}


\begin{frame}
    \frametitle{Le raisonnement par abduction}
    
\end{frame}


\begin{frame}
    \frametitle{Le raisonnement par induction}
    
\end{frame}


\begin{frame}
    \frametitle{Propriétées des différents types de raisonnement}
    \begin{center}
      %\begin{tabular}{ | l | c c c | }
      \begin{tabular}{ | p{0.14\linewidth} | p{0.24\linewidth} | p{0.24\linewidth} | p{0.24\linewidth} | }
        \hline \hline
         & Déduction & Abduction & Induction \\
        \hline \hline
        Procédure & Raisonnement à partir de prémisses pour aboutir à une conclusion via des règles \emph{formelles} d'inférence
                  & Raisonnement à partir d'observations incomplètes pour aboutir à \emph{l'explication}
                    la plus probable de ces observations
                  & Raisonnement à partir de maintes observations pour aboutir à une \emph{généralisation}
                    (que les prochaines observations seront semblables)
                  \\
        \hline
        Nécessaire & Oui (certain) & Non (incertain) & Non (incertain) \\
        \hline
        Ampliatif & Non & Oui & Oui \\
        \hline \hline
      \end{tabular}
    \end{center}
\end{frame}


%% END DOCUMENT
\end{document}

