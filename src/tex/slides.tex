%%%%%%%%%%%%%%%%%%%%%%%%%%%%%%%%%%%%%%%%%%%%%%%%%%%%%%
%% Slides : Introduction to probabilistic reasoning %%
%%%%%%%%%%%%%%%%%%%%%%%%%%%%%%%%%%%%%%%%%%%%%%%%%%%%%%

%% PREAMBLE
%% Define document class and basic options
\documentclass{beamer}
%\setlength{\parindent}{0pt}

%% Load packages
\usepackage{palatino}
\usepackage{amsfonts}
\usepackage{amsmath}
%\usepackage{url}
\usepackage{hyperref}
%\usepackage{listings}
\usepackage{verbatim}
\usepackage[utf8]{inputenc} %% For french
\usepackage{tikz} %% For drawing (eg, Venn diagrams)
\usefonttheme{serif}

\hypersetup{
	colorlinks=true,
	linkcolor=blue,
	citecolor=red,
	filecolor=blue,
	urlcolor=blue
}

\usetheme{Madrid}

%% Basic info
\title{Introduction à la pensée et aux méthodes en probabilité}
%\subtitle{}
\author{Roy Nitulescu\inst{1}}

\institute
{
    \inst{1}%
    CITADEL\\
    CR-CHUM
}

\date[UdeM, Sept. XX, 2023]{Université de Montréal, Sept. XX, 2023}

\AtBeginSection[]
{
    \begin{frame}
        \frametitle{Table des matières}
        \tableofcontents[currentsection]
    \end{frame}
}

\AtBeginSubsection[]
{
    \begin{frame}
        \frametitle{Table des matières}
        \tableofcontents[currentsubsection]
    \end{frame}
}


%% BEGIN DOCUMENT
\begin{document}

%%%%
%% Slides
%%%%

\frame{\titlepage}

\begin{frame}
    \frametitle{Épigraphe}
    ``La probabilité est le concept le plus important de la science moderne,
    d'autant plus que personne n'a la moindre idée de ce qu'elle signifie.'' -- Bertrand Russell, 1929
\end{frame}


\begin{frame}
    \frametitle{Préambule}
    
    Le code source pour cette présentation ce trouve ici:

    \vfill

    \url{https://github.com/rnitulescu/prob2023}
\end{frame}


\begin{frame}
    \frametitle{Table des matières}
    \tableofcontents
\end{frame}


%%%%
%% Le raisonnement
%%%%

\section{Le raisonnement}

\begin{frame}
    \frametitle{Le raisonnement}
    \textbf{Définition:}\

    \bigskip

    Le \emph{raisonnement} c’est la combinaison d’informations dans le
    but d’aboutir à une conclusion qui est liée à ces informations.\\

    \vfill \pause

    \textbf{Types de raisonnement:}\\

     par deduction, par abduction, par induction
\end{frame}


\begin{frame}
    \frametitle{Le raisonnement par déduction}
    \begin{itemize}
      \item Une approche \emph{formelle} de raisonnement \pause
      \item Commence avec des \emph{prémisses} \pause
      \item Le processus avance par l’application de règles formelle \emph{d’inférence} \pause
      \item Aboutit à une conclusion qui suit de façon \emph{nécessaire} des prémisses \pause
      \item Ne permet pas d’aboutir à des conclusions qui ne sont pas déjà
            implicites dans les prémisses (donc \emph{non-ampliatif})
    \end{itemize}
\end{frame}


\begin{frame}
    \frametitle{Exemples de déduction}
    \textbf{Ex 1:}\\ \pause
    \emph{Prémisse:} Tout les hommes sont mortels\\ \pause
    \emph{Prémisse:} Socrate est un homme\\ \pause
    \emph{Conclusion:} Socrate (étant un homme) est lui aussi mortel\

    \bigskip \pause

    \textbf{Ex 2:}\\
    Toute preuve mathématique
\end{frame}


\begin{frame}
    \frametitle{Le raisonnement par abduction}
    \begin{itemize}
      \item Approche \emph{informelle} de raisonnement \pause
      \item Commence avec un ensemble \emph{limité} d’observations \pause
      \item Le but est de trouver \emph{l’explication} la plus \emph{probable} des faits observés \pause
      \item La conclusion \emph{ne suit pas} de façon nécessaire des faits (donc incertain) \pause
      \item Permet d’aboutir à des explications qui ne sont pas
            implicites dans les faits (donc \emph{ampliatif})
    \end{itemize}
\end{frame}


\begin{frame}
    \frametitle{Exemples d'abduction}
    \textbf{Ex 1:}\\ \pause
    \emph{Fait:} Le sol est mouillé tout autour de ma maison\\ \pause
    \emph{Fait:} Le sol n'est pas mouillé autour des maisons de mes voisins\\ \pause
    \emph{Explications probables:}\\ \pause
          (1) Il y a eu de la pluie seulement au dessus de ma maison,\\ \pause
          (2) J'ai oublié d'éteindre mes arroseurs de jardin\\ \pause

    \smallskip

    \emph{Explication la plus probable:} J'ai oublié d'éteindre mes arroseurs de jardin\

    \bigskip \pause

    \textbf{Ex 2:}\\
    Travail de détectives policiers, diagnostics médicaux
\end{frame}


\begin{frame}
    \frametitle{Le raisonnement par induction}
    \begin{itemize}
      \item Approche \emph{informelle} de raisonnement \pause
      \item Commence avec un ensemble d’observations \pause
      \item Le but est de \emph{généraliser} à partir des faits observés \pause
      \item Le plus qu’on a d’observations fait sous des conditions homogènes,
            le plus probable nos conclusions \pause
      \item La conclusion \emph{ne suit pas} de façon nécessaire des faits (donc incertain) \pause
      \item Permet d’aboutir à des explications qui ne sont pas
            implicites dans les faits (donc \emph{ampliatif})
    \end{itemize}
\end{frame}


\begin{frame}
    \frametitle{Exemples d'induction}
    \textbf{Ex 1:}\\ \pause
    \emph{Observations:} Le soleil c’est levé et c’est couché à chaque jour de ma vie \pause
    \emph{Généralisation:} Le soleil ce lève et ce couche à chaque jour\

    \bigskip \pause

    \textbf{Ex 2:}\\
    Moyen d’apprentissage des enfants, études cliniques
\end{frame}


\begin{frame}
    \frametitle{Propriétées des différents types de raisonnement}
    \begin{center}
      %\begin{tabular}{ | l | c c c | }
      \begin{tabular}{ | p{0.14\linewidth} | p{0.24\linewidth} | p{0.24\linewidth} | p{0.24\linewidth} | }
        \hline \hline
         & Déduction & Abduction & Induction \\
        \hline \hline
        Procédure & Raisonnement à partir de prémisses pour aboutir à une conclusion via des règles \emph{formelles} d'inférence
                  & Raisonnement à partir d'observations incomplètes pour aboutir à \emph{l'explication}
                    la plus probable de ces observations
                  & Raisonnement à partir de maintes observations pour aboutir à une \emph{généralisation}
                    (que les prochaines observations seront semblables)
                  \\
        \hline
        Nécessaire & Oui (certain) & Non (incertain) & Non (incertain) \\
        \hline
        Ampliatif & Non & Oui & Oui \\
        \hline \hline
      \end{tabular}
    \end{center}
\end{frame}


%% END DOCUMENT
\end{document}

